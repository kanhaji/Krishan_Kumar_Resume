\documentclass[11pt,a4paper,sans]{moderncv} % Font sizes: 10, 11, or 12; paper sizes: a4paper, letterpaper, a5paper, legalpaper, executivepaper or landscape; font families: sans or roman

\moderncvstyle{casual} % CV theme - options include: 'casual' (default), 'classic', 'oldstyle' and 'banking'
\moderncvcolor{green} % CV color - options include: 'blue' (default), 'orange', 'green', 'red', 'purple', 'grey' and 'black'

\usepackage{lipsum} % Used for inserting dummy 'Lorem ipsum' text into the template
\usepackage{multicol}

\usepackage[scale=0.75]{geometry} % Reduce document margins

%----------------------------------------------------------------------------------------
%	NAME AND CONTACT INFORMATION SECTION
%----------------------------------------------------------------------------------------

\firstname{Krishan} % Your first name
\familyname{Kumar} % Your last name
% All information in this block is optional, comment out any lines you don't need
\title{Computer Science Engineering}
\address{H.N. 83 Sainik Vihar Colony, TCP3,Hisar Cantt,Hisar }{Haryana, 125006}
\mobile{(+91) 9671973851}
\email{kk14392r4@gmail.com}
\photo[70pt][0.6pt]{xc} % The first bracket is the picture height, the second is the thickness of the frame around the picture (0pt for no frame)
\quote{"Computers are like Old Testament gods; lots of rules and no mercy"}

%----------------------------------------------------------------------------------------

\begin{document}
\thispagestyle{empty}

\makecvtitle % Print the CV title

%----------------------------------------------------------------------------------------
%	EDUCATION SECTION
%----------------------------------------------------------------------------------------

\section{Education}
	\cventry{2015--Present}{Engineering}{Brcm College Of Engineering And Technology}{Bahal}{}{Computer Science Engineering}
	\cventry{2012--2013}{Intermediate}{Army Public School}{Hisar}{\textit{Percentage -- 72\%}}{ Maths,Physics and Chemistry}
	\cventry{2010--2011}{Xth Board}{Army Public School}{Hisar}{\textit{Percentage -- 73\%}}{Secondary School Certificate}
\section{Technical Skills}
	\begin{multicols}{3}
		\cvitem{}{C}
		\cvitem{}{$C^{++}$}
		\cvitem{}{Java}
		\cvitem{}{Python}
		\cvitem{}{Arduino}
		\cvitem{}{Algorithms}
		\cvitem{}{AVR Microcontroller}
		\cvitem{}{Basic Network Setup}
		\cvitem{}{PHP}
		\cvitem{}{HTML}
		\cvitem{}{Basic Linux}
\end{multicols}

\section{Soft Skills}
	\cvlistitem{Quick Learner.}
	\cvlistitem{Patience.}
	\cvlistitem{Adaptability}
	\cvlistitem{Able to handle multiple situations at the same time.}
	\cvlistitem{Creating Ideas: Creativity.}
	\cvlistitem{Can manage time effectively}
	\cvlistitem{Good Computer handling skills.}
	\cvlistitem{Team Work.}
	
\section{Projects}
	\cventry{Feb 2018}{Voip Server}{Just For Fun. This server is still under Development. To make free audio and video call on College's Intranet Free Of Cost without worring about network coverage and data charges}{}{}{}
	\cventry{Sep 2017}{Mailing Server}{To Maintain Authenticity and unity in Events on below server a Registration Process was also made mandatory}{To achieve that this server was required}{}{}
	\cventry{Sep 2017}{Web Server}{Http Server For Organizing Online Events Like Quizs and Apptitude Test in college during Society Events}{}{}{}
	\cventry{July 2017--Aug 2017}{Lab Assistant}{A Java Based Software to help lab assistants in installing one software on many computers on LAN from one System}{Based on Telnet,Ftp,Windows Registry editing}{}{}
	\cventry{Jul 2017--Aug 2017}{Home Automation}{Basic IOT enabled Home Automation Project with 3 MCU boards Connected with 5 relay siwtches each and communicating over WI-FI}{Master Slave Model}{}{} 

	\cventry{Jun 2017-Jul 2017}{Line Follower Updated}{Updated Previous line follower system to Work in Dual mode remote controlled and line follower Remote is a self developed Android App.}{\textit{Added Bluetooth module to communicate with smart Phone}}{$http://youtu.be/j_2RbmJ2vj4$}{}
	
	\cventry{May 2017 - Jun 2017}{Line Follower}{1st Project With 7 IR array white line follower. PID controlled}{}{}{}
	


\end{document}
